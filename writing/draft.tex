\documentclass{article}
\usepackage{setspace}
\usepackage[T1]{fontenc}
\usepackage{amsmath}
\usepackage{graphicx}
\usepackage{subcaption}
\usepackage[a4paper,margin=2.5cm]{geometry}
\usepackage{float}

\usepackage[hidelinks]{hyperref}
\usepackage[nameinlink,noabbrev]{cleveref}

\usepackage{siunitx}
\sisetup{
  detect-all,
  per-mode=symbol,        % z.B. g/mol statt g mol^{-1}
  exponent-product = \cdot,
  separate-uncertainty,
  % Wähle eine Locale:
  % locale = DE,          % Komma als Dezimaltrenner
  locale = US,            % Punkt als Dezimaltrenner (englischer Text)
}

\crefname{equation}{Eq.}{Eqs.}
\Crefname{equation}{Equation}{Equations}
\crefname{figure}{Fig.}{Figs.}
\Crefname{figure}{Figure}{Figures}
\crefname{subfigure}{Fig.}{Figs.}
\Crefname{subfigure}{Figure}{Figures}
\crefname{table}{Table}{Tables}
\Crefname{table}{Table}{Tables}
\crefname{section}{Sec.}{Secs.}
\Crefname{section}{Section}{Sections}

\setlength{\parindent}{0pt} 
\setlength{\parskip}{10pt} 

\usepackage{xcolor}
\usepackage{listings}

\lstdefinestyle{code}{
  basicstyle=\ttfamily\small,
  backgroundcolor=\color{gray!10},
  frame=single,
  numbers=left,
  numberstyle=\tiny,
  keywordstyle=\color{blue},
  commentstyle=\color{green!50!black}\itshape,
  stringstyle=\color{red!60!black},
  showstringspaces=false,
  tabsize=2,
  breaklines=true,
  captionpos=b
}

\newcommand{\code}[1]{\texttt{#1}}
\lstset{style=code}

\begin{document}
\onehalfspacing

\begin{titlepage}
\begin{center}
 
\Large\textbf{Department of Physics and Astronomy\\
Heidelberg University}

\vspace{18cm}

\normalsize
Bachelor Thesis in Physics\\
submitted by\\
\vspace{0.5cm}
\Large\textbf{Gabriel Scherer}\\
\normalsize
\vspace{0.5cm}
born in Mainz (Germany)\\
\vspace{0.5cm}
\Large\textbf{2003}
\normalsize

\newpage




\Large\textbf{About ...}

\vspace{18cm}

\normalsize
This Bachelor Thesis has been carried out by Gabriel Scherer at the\\
Institute for theoretical Physics in Heidelberg\\
under the supervision of\\
Prof. Tristan Bereau

\vfill
\end{center}

\pagebreak
\tableofcontents

\end{titlepage}


\pagebreak

\begin{abstract}
    
\end{abstract}

\section{Introduction}

Coarse-grained (CG) molecular dynamic (MD) models are a very 
effective to simulate large molecular systems. Unlike all atom simulation 
which aim to simulate the trajectories of every atom, the idea is to 
substitute groups of atoms by a bead which is given properties that match 
the corresponding atoms. 

Using this CG approach has the benefits of much lower 
computational costs which makes it viable to simulate large systems that
would be unattainable for a all atoms simulation. The most popular method
for CG molecular dynamics is the MARTINI model with the 
Lennard-Jones potential as the underlying force-field. 

In this paper I will use the MARTINI method as a baseline for a system but
then apply a different force-field \textbf{dissipative particle dynamics}
(DPD). To test this approach I will monitor the self-assembly of 
a lipid bilayer and afterwards test the mechanical properties, more 
specifically the bending modulus.

Lipid bilayers are a very important biological structure due to
it being the basis of almost all membranes in cells. The polar heads
keep any polar or charged particles or molecules out thus
making it a very effective structure and worth exploring with DPD.
The chosen lipid 1-palmitoyl-2-oleoyl-sn-glycero-3-phosphocholine 
(POPC) consist of a choline head group which is connected with the 
oleoyl and palmitoyl tails by a phosphate and glycerol group. 

The simulation will be carried using the LAMMPS software which is a very
flexible molecular dynamics simulator.

\section{Methods}

\subsection{Dissipative particle dynamics}

For DPD the idea is to find a force which follows the Langevin equation.
The force on a particle $i$ by other particles $j$ is:

\begin{equation}
    m \frac{\text{d}\mathbf{v}_i}{\text{d}t} = \sum_{j \neq i} \mathbf{F}^C_{ij} + 
    \mathbf{F}^D_{ij} + \mathbf{F}^R_{ij}.  
\end{equation}

There are three different components to the force, namely the conservative
force $F^C$, the dissipative Force $F^D$ and the random force $F^R$. 

The conservative force can be expressed using $r_c$ as a cutoff distance, 
a particle pair specific parameter $A_{ij}$, the distance vector
$\mathbf{r}_{ij} = \mathbf{r_i} - \mathbf{r_j}$, the distance $r_{ij}$ = $|\mathbf{r}_{ij}|$
and the normalized vector $\hat{\mathbf{r}}_{ij} = \mathbf{r}_{ij}/r_{ij}$ as:

\begin{equation}
    \mathbf{F}^C_{ij} = \begin{cases}
        A_{ij}\left (1 - \frac{r_{ij}}{r_c} \right )\hat{\mathbf{r}}_{ij}, & r_{ij} \leq r_c \\
        0, & r_{ij} > r_c 
    \end{cases}
    . 
\end{equation}

Important to note is that the cutoff distance is usually set to $r_c = 1$ 
making it redundant in the formula above. Furthermore a big difference is that
DPD does not postulate a potential which results in a force but rather starts
with said force. A useful comparison for this is the Lennard-Jones potential
which is used in MARTINI simulations and works the other way around.

\begin{equation}
    U^{LJ} = 4 \epsilon \left [\left(\frac{\sigma}{r} \right)^{12} 
    - \left(\frac{\sigma}{r} \right)^6 \right ]. \label{LJ_pot}
\end{equation}

\begin{figure}[htbp]
  \centering
  \begin{subfigure}{0.48\textwidth}
    \centering
    \includegraphics[width=\linewidth]{figures/dpd_force_potential.png} 
    \caption{DPD force and potential}
    \label{fig:sub1}
  \end{subfigure}\hfill
  \begin{subfigure}{0.48\textwidth}
    \centering
    \includegraphics[width=\linewidth]{figures/lj_force_potential.png} 
    \caption{Lennard-Jones force and potential}
    \label{fig:sub2}
  \end{subfigure}
  \caption{Comparison between DPD and MARTINI force}
  \label{comp_forces}
\end{figure}

As seen in \cref{comp_forces} there are key differences between both 
approaches. Most notably the DPD force and the corresponding potential
do not diverge at $r=0$ which means that overlapping particles are allowed
making it a soft potential. Adding to this DPD is a purely repulsive 
force whereas Lennard-Jones has a clear attractive part and a minimum.

The dissipative and random force can best be described by using two weight 
functions $w^D$ and $w^R$. Using the work of Espanol and Warren (\textbf{cite here})
one weight functions can be arbitrarily but in turn fixes the other one. 
The usual choice is:

\begin{equation}
    w^D(r_{ij}) = \left [w^R(r_{ij}) \right ]^2 = \begin{cases}
        \left (1 - \frac{r_{ij}}{r_c} \right )^2, & r_{ij} \leq r_c \\
        0, & r_{ij} > r_c
    \end{cases}
\end{equation}

One also need two strength coefficients, namely $\sigma$ for the random force
and $\gamma$ for the drag force. The fluctuation dissipation theorem connects these
two parameters with the thermal Energy $k_B T$ and yields:

\begin{equation}
    \sigma^2 = 2 \gamma k_B T.
\end{equation}

Defining $\mathbf{v}_{ij} = |\mathbf{v}_i - \mathbf{v}_j|$ and $\alpha_{ij}$ as 
a random number with zero mean and unit variance 
results in a final expression for both forces:

\begin{align}
    \mathbf{F}^D_{ij} &= -\gamma w^D(r_{ij})(\hat{\mathbf{r}}_{ij}\cdot \mathbf{v}_{ij})
    \hat{\mathbf{r}}_{ij} \\
    \mathbf{F}^R_{ij} &= \sigma w^R(r_{ij}) \alpha_{ij} \Delta t^{-1/2} \hat{\mathbf{r}}_{ij}.
\end{align}

Notably a factor of $\Delta t^{-1/2}$ appears in the random force. This
follows the argument of Groot and Warren (\textbf{cite here}) as this 
corrects the time dependency of displacement to the known expression 
$\left < x^2\right >\propto t$.




\subsection{Coarse-grained simulations}\label{CG_models}

The most important part of the CG models is the representation of 
different groups of atoms. In MARTINI there are five different basic bead 
types: water (W), polar (P), non-polar (N), a-polar (C), charged (Q).
The P and C beads are also given a number from 1 to 5 depending on the 
polarity, reaching from P1 being the least polar and P5 being the most
polar and C1 being the most a-polar and C5 the least a-polar.
The W bead is identical to the P4 bead. The Q and N bead can also have 
subtypes which quantify the likeliness of hydrogen bond forming capabilities.
There is 0 meaning no capabilities, d meaning some donor capabilities and 
a meaning some acceptor capabilities.  

Applying this to the POPC lipid we have the Q0 choline head group, the Qa
phosphate group, the two Na glycerol groups which connect to the two tails
consisting of respectively 4 C1 or 3 C1 and 1 C3. This can be compared
to \cref{popc_lammps} where the Q0 choline head group is red, the 
phosphate Qa group is blue, the two glycerol groups are pink and the tails 
are black with the singular C3 bead being silver.

\begin{figure}[H]
\centering
\includegraphics[width=10cm]{figures/popc_cg.png}
\caption{POPC molecule in LAMMPS}
\label{popc_lammps}
\end{figure}

One thing to note is that there are two charged beads but the DPD force does
not contain any distinct interaction between them so this has to be added
when simulating this system. Crucially one can not simply take normal 
coulomb interaction as it would destroy the soft interaction. To solve this
I chose a charge distribution which corresponds to a exponentially smeared
out charge. This is following the work of wang and hernandez (\textbf{cite here})
who also introduce a factor $\beta$ which controls the strength of the
smearing effect. Using earlier works from Gonzales-Melchor et al (\textbf{cite here})
this factor and its inverse $\lambda$ which is used in the LAMMPS implementation 
of the potential was calculated as:

\begin{equation}
    \beta = 0.844 \Longrightarrow \lambda = 1/\beta = 1.185.
\end{equation}

With this and a conversion constant $C$ the coulomb interaction 
follows the following formulas:

\begin{align}
    \rho(r) = \frac{q \beta^2}{\pi r} \Longrightarrow
    \mathbf{F}^E_{ij} = \frac{C q_i q_j}{4 \pi r_{ij}}[1 - \exp(-2\beta r_{ij})
    (1+2\beta r_{ij}(1+\beta r_{ij}))]\hat{\mathbf{r}}_{ij}.
\end{align}

With the DPD forces and the now introduced coulomb force every unbounded interaction 
can be calculated. The next step is to choose potentials which govern the dynamics 
inside of a molecule. Here the POPC molecule only has two other forms of connection,
namely the bonds between two beads and the angle between three beads. For the bond 
potential the easiest choice is a harmonic potential

\begin{equation}
    E_{ij} = K_b(r_{ij} - r_0)^2 \label{harmonic}
\end{equation}

where $K_b$ is a spring constant which includes the usual factor $1/2$ and $r_0$ is 
the equilibrium bond distance. For the angle between three atoms I also chose the potential

\begin{equation}
    E = K_a(\theta - \theta_0)^2
\end{equation}

with $K_a$ as an energy constant similarly to before and $\theta_0$ as the equilibrium
angle. One could also choose a cosine potential for the angle as well but that brings a
disadvantage at $\theta_0 = 180^\circ$. The slope around that value is extremely small 
resulting in a vanishing force. 

\subsection{DPD units}\label{dpd_units}

When doing molecular dynamics simulations it is very common to work with reduced units.
Meaning there are reference parameters which are used to make every physical quantity
unitless. For MARTINI these are usually the parameters of the Lennard-Jones potential
$\epsilon$ and $\sigma$ as in \cref{LJ_pot}. But since DPD has a very different 
force and thus potential one needs different reference parameters. For the 
length parameter I follow the work of wang and hernandez (\textbf{cite here}) who 
choose the length scale as the length of a cube containing three CG water beads where each 
corresponds to four water molecules $N_w = 4$. The earlier work of Groot and Warren 
(\textbf{cite here}) used the number density $\rho = 3$ beads per unit of volume. 
Lastly the volume of one water molecule $V_w$ can be calculated by
using the molar mass $M=18.02 \si{\g\per\mol}$,
the density $\rho_w = 1 \si{\g\per\cubic\centi\meter}$ and the avogadro
 constant $N_A = 6.022 \cdot 10^{23} 
\text{molecules}/\text{mol}$

\begin{equation}
    V_w = \frac{M}{\rho_w N_A} = 2.99 \cdot 10^{-23} \si{\cubic\centi\m} 
    = 29.9 \si{\cubic\angstrom}.
\end{equation}

The DPD length follows directly

\begin{equation}
    r_{\text{ref}} = \sqrt[3]{\rho \cdot N_w \cdot V_w} = 7.11 \si{\angstrom}.
\end{equation}

For an energy scale the chosen value is much simpler. I simply took the temperature 
$T_{\text{ref}}=300 \si{\K}$ and used the thermal energy as a reference resulting in 

\begin{equation}
    \epsilon_{\text{ref}} = k_{\mathrm{B}} T_{\text{ref}} = 4.14 \cdot 10^{-21} \si{\J} = 
    25.8 \si{\milli\eV}. 
\end{equation}

\subsection{Parametrization}

There are multiple parameters that need to be set in this system. The first are the 
parameters for the bonds and angles. For this the MARTINI database is used.
There the POPC molecule has a corresponding \path{martini_v_2.0_POPC.itp} file which
specifies the bead types discussed in \cref{CG_models}, bead charge and also angles and bonds 
which connect the different beads. The POPC molecule contains twelve beads with six different bead types,
eleven bonds with two bond types and ten angles with three angle types. 

The two bond types are parametrized with 

\[
r_1 = 0.47 \si{\nano\m}, \quad k_1 = 1250 \si{\kilo\J\per\mol\per\nano\m\squared} 
\]
\[
r_2 = 0.37 \si{\nano\m}, \quad k_2 = 1250 \si{\kilo\J\per\mol\per\nano\m\squared}
\]

For the bond length we can simply divide by our reference length from 
\cref{dpd_units}. The energy constant is a bit more complicated. Important 
to note is that the MARTINI definition still has a factor $1/2$ in front of 
the harmonic potential which is absorbed in the LAMMPS implementation, 
see \cref{harmonic}. Due to this we calculate 

\begin{equation}
    1250 \si{\kilo\J\per\mol\per\nano\m\squared}\cdot \frac{1}{2} \cdot
    \frac{1000\si{J}}{1 \si{\kilo\J}} \cdot
    \frac{\si{\mol}}{6.022 \cdot 10^{23}} \cdot 
    \frac{1}{4.14 \cdot 10^{-21}\si{J}} \cdot
    (0.711 \si{\nano\m})^2 = 126.7
\end{equation}

This leads to our parameters in reduced units which are from 
now on written with a hat:

\[
\boxed{
\begin{aligned}
\hat{r}_1 &= 0.661, \quad \hat{k}_1 = 126.7\\
\hat{r}_2 &= 0.520, \quad \hat{k}_2 = 126.7
\end{aligned}
}
\]

For the angle coefficients we again start from the MARTINI data.

\[
\theta_1 = 180^\circ, \quad k^\theta_1 = 25 \si{\kilo\J\per\mol}
\]
\[
\theta_2 = 120^\circ, \quad k^\theta_2 = 25 \si{\kilo\J\per\mol}
\]
\[
\theta_3 = 120^\circ, \quad k^\theta_3 = 45 \si{\kilo\J\per\mol}
\]


We still have to use factor $1/2$ similar to the bond calculation. 
The process is the same for every $i \in 
\{1,2,3\}$

\begin{equation}
    \{25, 45\} \si{\kilo\J\per\mol} \cdot \frac{1}{2} \cdot \frac{1000 \si{J}}{1 \si{\kilo\J}}
    \cdot \frac{\si{\mol}}{6.022 \cdot 10^{23}} \frac{1}{4.14 \cdot 10^{-21} \si{\J}}  = 
    \{ 5.01, 9.02 \}
\end{equation}

With this there are the final parameters for the angle potential in the POPC molecule. 

\[
\boxed{
\begin{aligned}
\hat{\theta}_1 &= 180^\circ, \quad \hat{k}^\theta_1 = 5.01\\
\hat{\theta}_2 &= 120^\circ, \quad \hat{k}^\theta_2 = 5.01\\
\hat{\theta}_3 &= 120^\circ, \quad \hat{k}^\theta_3 = 9.02
\end{aligned}
}
\]

The final step are the parameters for the DPD force which determine the maximum repulsion 
at $r_{ij} =0$. This was taken from Wang and Hernandez (\textbf{cite here}). 

\begin{table}[H]
\centering
\caption{DPD interaction parameters $a_{ij}$ and cutoff radius $r_c$ for different bead pairs}
\label{tab:dpd_params}
\begin{tabular}{lccc}
\hline
Bead pair & $A_{ij}$ & $\gamma$ & Charge \\
\hline
W - W      & 102.067 & 4.5 & 0 \\
W - C1     & 130.616 & 4.5 & 0 \\
W - C3     & 121.734 & 4.5 & 0 \\
W - Na     & 108.411 & 4.5 & 0 \\
W - Q0     & 84.938  & 4.5 & 0 \\
W - Qa     & 94.454  & 4.5 & 0 \\
Qa - Qa    & 102.067 & 4.5 & $-1$ \\
Qa - C1    & 130.616 & 4.5 & 0 \\
Qa - C3    & 126.809 & 4.5 & 0 \\
Qa - Na    & 108.411 & 4.5 & 0 \\
Qa - Q0    & 98.895  & 4.5 & $+1$ \\
Q0 - Q0    & 102.067 & 4.5 & $+1$ \\
Q0 - C1    & 121.099 & 4.5 & 0 \\
Q0 - C3    & 117.293 & 4.5 & 0 \\
Q0 - Na    & 98.895  & 4.5 & 0 \\
Na - Na    & 102.067 & 4.5 & 0 \\
Na - C1    & 115.390 & 4.5 & 0 \\
Na - C3    & 115.390 & 4.5 & 0 \\
C3 - C3    & 102.067 & 4.5 & 0 \\
C3 - C1    & 102.067 & 4.5 & 0 \\
C1 - C1    & 102.067 & 4.5 & 0 \\
\hline
\end{tabular}
\end{table}
























\end{document}
