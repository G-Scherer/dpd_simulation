\documentclass{article}
\usepackage{setspace}
\usepackage[T1]{fontenc}
\usepackage{amsmath}

\setlength{\parindent}{0pt} %Einrückung von Absätzen auf null gesetzt
\setlength{\parskip}{10pt} %Abstand zischen Absätzen auf 10pt gesetzt

\begin{document}
\onehalfspacing

\begin{titlepage}
\begin{center}
 
\Large\textbf{Department of Physics and Astronomy\\
Heidelberg University}

\vspace{18cm}

\normalsize
Bachelor Thesis in Physics\\
submitted by\\
\vspace{0.5cm}
\Large\textbf{Gabriel Scherer}\\
\normalsize
\vspace{0.5cm}
born in Mainz (Germany)\\
\vspace{0.5cm}
\Large\textbf{2003}
\normalsize

\newpage




\Large\textbf{About ...}

\vspace{18cm}

\normalsize
This Bachelor Thesis has been carried out by Gabriel Scherer at the\\
Institute for theoretical Physics in Heidelberg\\
under the supervision of\\
Prof. Tristan Bereau

\vfill
\end{center}

\pagebreak
\tableofcontents

\end{titlepage}


\pagebreak

\begin{abstract}
    
\end{abstract}

\section{Introduction}

Coarse-grained (CG) molecular dynamic (MD) models are a very 
effective to simulate large molecular systems. Unlike all atom simulation 
which aim to simulate the trajectories of every atom, the idea is to 
substitute groups of atoms by a bead which is given properties that match 
the corresponding atoms. 

Using this CG approach has the benefits of much lower 
computational costs which makes it viable to simulate large systems that
would be unattainable for a all atoms simulation. The most popular method
for CG molecular dynamics is the MARTINI model with the 
Lennard-Jones potential as the underlying force-field. 

In this paper I will use the MARTINI method as a baseline for a system but
then apply a different force-field \textbf{dissipative particle dynamics}
(DPD). To test this approach I will monitor the self-assembly of 
a lipid bilayer and afterwards test the mechanical properties, more 
specifically the bending modulus.

Lipid bilayers are a very important biological structure due to
it being the basis of almost all membranes in cells. The polar heads
keep any polar or charged particles or molecules out thus
making it a very effective structure and worth exploring with DPD.
The chosen lipid 1-palmitoyl-2-oleoyl-sn-glycero-3-phosphocholine 
(POPC) consist of a choline head group which is connected with the 
oleoyl and palmitoyl tails by a phosphate and glycerol group. 

The simulation will be carried using the LAMMPS software which is a very
flexible molecular dynamics simulator.

\section{Methods}

\subsection{Dissipative particle dynamics}

For DPD the idea is to find a force which follows the Langevin equation.
The force on a particle $i$ by other particles $j$ is:

\begin{equation}
    m \frac{\text{d}v_i}{\text{d}t} = \sum_{j \neq i} F^C_{ij} + 
    F^D_{ij} + F^R_{ij}.  
\end{equation}

There are three different components to the force, namely the conservative
force $F^C$, the dissipative Force $F^D$ and the random force $F^R$. All 
forces depend on a weight function:

\begin{equation}
    w(r_{ij}) = 1 - r_{ij}/r_c.
\end{equation}



The conservative part is a simple linear repulsion with a constant
which depends on the particle pair $i,j$:




\subsection{Coarse-grained methods}

The most important part of the CG models is the representation of 
different groups of atoms. In MARTINI there are five different basic bead 
types: water (W), polar (P), non-polar (N), a-polar (C), charged (Q).
The P and C beads are also given a number from 1 to 5 depending on the 
polarity, reaching from P1 being the least polar and P5 being the most
polar and C1 being the most a-polar and C5 the least a-polar.
The W bead is identical to the P4 bead. The Q and N bead can also have 
subtypes which quantify the likeliness of hydrogen bond forming capabilities.
There is 0 meaning no capabilities, d meaning some donor capabilities and 
a meaning some acceptor capabilities.  

Applying this to the POPC lipid we have the Q0 choline head group, the Qa
phosphate group, the two Na glycerol groups which connect to the two tails
consisting of respectively 4 C1 or 3 C1 and 1 C3. 














\end{document}
