% --- 1. DOKUMENTENKLASSE & GLOBALE OPTIONEN ---
\documentclass[
    11pt,
    a4paper,
    oneside,        % Zweiseitig für Buchdruck
    BCOR=10mm,      % Bindekorrektur (Platz für die Klebebindung)
    parskip=half,   % Ersetzt manuelles setlength für Absätze (Europäischer Standard)
    listof=totoc,   % Abbildungs-/Tabellenverzeichnis ins Inhaltsverzeichnis aufnehmen
    bibliography=totoc % Literaturverzeichnis ins Inhaltsverzeichnis
]{scrreprt}

% --- 2. SPRACHE & ENCODING ---
\usepackage[utf8]{inputenc} % Nur nötig bei älteren LaTeX Versionen, schadet aber nicht
\usepackage[T1]{fontenc}    % Wichtig für korrekte Umlaute/Sonderzeichen im Output
\usepackage[english]{babel} % WICHTIG: Korrekte Silbentrennung für Englisch

% --- 3. SCHRIFTARTEN & MATHE ---
\usepackage{mathpazo}       % Palatino (Serifen) für Text und Mathe
\usepackage[scaled=.95]{helvet} % Helvetica (Serifenlos) für Überschriften
\usepackage{courier}        % Courier für Code
\usepackage{amsmath}        % Erweiterte Mathe-Umgebungen
\usepackage{setspace}       % Für Zeilenabstände (z.B. \onehalfspacing)

% --- 4. GRAFIKEN & TABELLEN ---
\usepackage{graphicx}
\usepackage{subcaption}     % Für Unter-Abbildungen (Bild a, Bild b)
\usepackage{float}          % Für erzwungene Positionierung [H] (sparsam nutzen!)

% Optional: Wenn du wirklich feste 2.5cm Ränder brauchst (sonst auslassen für besseren Look):
% \usepackage[a4paper,margin=2.5cm]{geometry}

% --- 5. LAYOUT & KOPF-/FUSSZEILEN ---
% --- LAYOUT & KOPFZEILEN ---
\usepackage[automark, headsepline]{scrlayer-scrpage}
\clearpairofpagestyles 

% Definition für NORMALE Seiten (scrheadings):
\ohead{\pagemark}   % Seitenzahl oben außen
\ihead{\headmark}   % Kapitelbezeichnung oben innen

% Definition für KAPITEL-STARTSEITEN (plain):
% Der Befehl in den eckigen Klammern [] gilt nur für den "Plain"-Stil (Kapitelanfang)
\cfoot[\pagemark]{} % Seitenzahl unten mittig bei neuen Kapiteln

\automark[section]{chapter} 
\setkomafont{pageheadfoot}{\normalfont\sffamily}
\setkomafont{pagination}{\normalfont\sffamily}

% --- RÄNDER (Dein Wunsch: 2.5cm) ---
% Wir nutzen geometry, um die Ränder festzusetzen
\usepackage[a4paper,margin=2.5cm,bindingoffset=5mm]{geometry}

% --- 6. PHYSIK & EINHEITEN (siunitx) ---
\usepackage{siunitx}
\sisetup{
  detect-all,             % Passt sich der Schriftart an (fett, kursiv etc.)
  per-mode=symbol,        % m/s statt m s^{-1}
  exponent-product = \cdot,
  separate-uncertainty,
  locale = US,            % Punkt als Dezimaltrenner
  range-phrase = --,      % Strich für Bereiche (1--10 nm)
  range-units = single    % 1--10 nm statt 1 nm -- 10 nm
}

% --- 7. CODE DARSTELLUNG ---
\usepackage{xcolor}
\usepackage{listings}
\lstdefinestyle{code}{
  basicstyle=\ttfamily\small,
  backgroundcolor=\color{gray!10},
  frame=single,
  numbers=left,
  numberstyle=\tiny\color{gray},
  keywordstyle=\color{blue!80!black},
  commentstyle=\color{green!40!black}\itshape,
  stringstyle=\color{red!60!black},
  showstringspaces=false,
  tabsize=2,
  breaklines=true,
  captionpos=b
}
\lstset{style=code}
\newcommand{\code}[1]{\texttt{#1}}

% --- 8. REFERENZEN & LINKS (Muss am Ende stehen) ---
\usepackage[hidelinks]{hyperref} % Klickbare Links, aber keine roten Kästen

% Cleveref: Automatische Namen (Fig. 1, Eq. 2)
% "noabbrev" entfernt, da du unten Abkürzungen definiert hast.
\usepackage[nameinlink]{cleveref} 

% Deine Definitionen für Referenzen (Physik-Standard ist oft abgekürzt):
\crefname{equation}{Eq.}{Eqs.}
\Crefname{equation}{Equation}{Equations}
\crefname{figure}{Fig.}{Figs.}
\Crefname{figure}{Figure}{Figures}
\crefname{subfigure}{Fig.}{Figs.}  % Subfigures auch als Fig. referenzieren
\crefname{table}{Table}{Tables}
\crefname{section}{Sec.}{Secs.}
\Crefname{section}{Section}{Sections}

% --- EINSTELLUNGEN KOMA-SCRIPT ---
% Captions verschönern (Fett und etwas kleiner)
\setkomafont{captionlabel}{\bfseries}
\setkomafont{caption}{\small}

\begin{document}
\onehalfspacing

\begin{titlepage}
\pagestyle{plain}
\pagenumbering{roman}

\begin{center}
 
\Large\textbf{Department of Physics and Astronomy\\
Heidelberg University}

\vspace{18cm}

\normalsize
Bachelor Thesis in Physics\\
submitted by\\
\vspace{0.5cm}
\Large\textbf{Gabriel Scherer}\\
\normalsize
\vspace{0.5cm}
born in Mainz (Germany)\\
\vspace{0.5cm}
\Large\textbf{2003}
\normalsize

\newpage




\Large\textbf{About ...}

\vspace{18cm}

\normalsize
This Bachelor Thesis has been carried out by Gabriel Scherer at the\\
Institute for theoretical Physics in Heidelberg\\
under the supervision of\\
Prof. Tristan Bereau

\vfill
\end{center}

\end{titlepage}

\chapter*{Abstract}

\tableofcontents

\clearpage

\pagenumbering{arabic}
\pagestyle{scrheadings}

\chapter{Introduction}

Coarse-grained (CG) molecular dynamic (MD) models are a very 
effective method to simulate large molecular systems. In contrast to all atom simulation 
which resolve individual atomistic trajectories, CG models represent atomic groups
as interaction sites or beads.

The reduced complexity of the system offers substantial acceleration compared to the atomistic 
approach making it possible to simulate much larger systems which are unattainable in all 
atom simulations. The most popular method
for CG molecular dynamics is the MARTINI model with the 
Lennard-Jones potential as the underlying force-field. 

This thesis employs the MARTINI model as a baseline for a system but
then applies a different force-field, namely \textbf{dissipative particle dynamics}
(DPD). To test this approach the self-assembly of 
a lipid bilayer is monitored and afterwards the mechanical properties are determined, more 
specifically the bending modulus.

Lipid bilayers are a fundamental biological structure as they constitute 
the basis of almost all membranes in cells. The polar heads act as a barrier for
any polar or charged particles and molecules thus
making it a very effective structure and worth exploring with DPD.
The chosen lipid 1-palmitoyl-2-oleoyl-sn-glycero-3-phosphocholine 
(POPC) consists of a choline head group which is connected with the 
oleoyl and palmitoyl tails by a phosphate and glycerol group. 

The simulations were carried out using the LAMMPS software which is a
versatile molecular dynamics simulation package.

\chapter{Methods}

\section{Coarse-grained simulations}\label{CG_models}

Central to CG models is the representation of 
different groups of atoms as beads. MARTINI distinguishes between five basic bead 
types: water (W), polar (P), non-polar (N), apolar (C), charged (Q).
The P and C beads are also assigned a number from 1 to 5 depending on the 
polarity, reaching from P1 being the least polar and P5 being the most
polar whereas C1 is the most apolar and C5 the least apolar.
The W bead is identical to the P4 bead. The Q and N beads can also have 
subtypes which quantify the hydrogen bond forming capabilities.
There is 0 meaning no capabilities, d meaning some donor capabilities and 
a meaning some acceptor capabilities.  

In the case of the POPC lipid this results in a charged Q0 choline head group, a Qa
phosphate group and two Na glycerol groups which connect to the two tails. These
consist of 4 C1 or 3 C1 and 1 C3, respectively. The resulting molecule 
is shown in \cref{popc_lammps}.

\begin{figure}[t]
\centering
\includegraphics[width=10cm]{figures/popc_cg.png}
\caption{Coarse-grained POPC molecule representation in LAMMPS. The Q0 choline head group is red, 
the Qa phosphate group is blue,
the connecting Na glycerol groups is violet and the apolar tails are made up of the
black C1 beads and the silver C3 bead.}
\label{popc_lammps}
\end{figure}

The topology of the molecule is defined in the \path{martini_v_2.0_POPC.itp} file 
which can be downloaded from the MARTINI website. It serves as the baseline for the 
parametrization in the following sections. It lists the employed potential for the 
bonds and angles, including their equilibrium parameters in MARTINI units, as well
as the bead types used in the representation. 


\section{Dissipative particle dynamics} \label{dpd}

The equations of motion for a particle $i$ influenced by all other particles $j$
in a DPD force field is

\begin{equation}
    m \frac{\text{d}\mathbf{v}_i}{\text{d}t} = \sum_{j \neq i} \mathbf{F}^C_{ij} + 
    \mathbf{F}^D_{ij} + \mathbf{F}^R_{ij}\, ,  
\end{equation}

containing three different parts, more specifically the conservative
force $\mathbf{F}^C$, the dissipative force $\mathbf{F}^D$ and the random force $\mathbf{F}^R$. 

The conservative force can be expressed as

\begin{equation}
    \mathbf{F}^C_{ij} = \begin{cases}
        A_{ij}\left (1 - \frac{r_{ij}}{r_c} \right )\hat{\mathbf{r}}_{ij}, & r_{ij} \leq r_c \\
        0, & r_{ij} > r_c 
    \end{cases}
    \, ,
\end{equation}

where $r_c$ is a cutoff distance and $A_{ij}$ is
a particle pair specific repulsion parameter. Furthermore,
$\mathbf{r}_{ij} = \mathbf{r}_i - \mathbf{r}_j$ defines the distance vector,
with $r_{ij} = |\mathbf{r}_{ij}|$ being its magnitude 
and $\hat{\mathbf{r}}_{ij} = \mathbf{r}_{ij}/r_{ij}$ the corresponding unit vector .

Usually the cutoff distance is set to $r_c = 1$ when working with reduced units, which will be 
discussed in \cref{dpd_units}, 
making it redundant in the formula above. DPD distinguishes itself from MARTINI in not
postulating a potential which results in a force field but it employs a force field without 
starting from a fundamental potential. To further illustrate this the Lennard-Jones potential 

\begin{equation}
    U^{LJ} = 4 \epsilon \left [\left(\frac{\sigma}{r} \right)^{12} 
    - \left(\frac{\sigma}{r} \right)^6 \right ]\, , \label{LJ_pot}
\end{equation}
with its parameters $\sigma$ and $\epsilon$ depending on the particles, which is
commonly used in the MARTINI method is compared to the DPD force field in \cref{comp_forces}.
The key difference between both 
approaches is that the DPD force and the corresponding potential
do not diverge at $r=0$ which means that overlapping particles are allowed
making it a soft potential. Adding to this DPD is a purely repulsive 
force whereas Lennard-Jones has a clear attractive part and a minimum.

\begin{figure}[t]
  \centering
  \begin{subfigure}{0.48\textwidth}
    \centering
    \includegraphics[width=\linewidth]{figures/dpd_force_potential.png} 
    \caption{}
    \label{fig:sub1}
  \end{subfigure}\hfill
  \begin{subfigure}{0.48\textwidth}
    \centering
    \includegraphics[width=\linewidth]{figures/lj_force_potential.png} 
    \caption{}
    \label{fig:sub2}
  \end{subfigure}
  \caption{(a) The DPD force and its corresponding potential. (b) The Lennard-Jones 
  potential used in the MARTINI method. Key difference is the DPD force and potential 
  do not diverge at the pair distance $r_{ij}=0$. }
  \label{comp_forces}
\end{figure}

The dissipative and random force can best be described by using two weight 
functions $w^D$ and $w^R$. Using the work of Espanol and Warren (\textbf{cite here})
one weight functions can be arbitrarily chosen but in turn fixes the other one. 
The usual choice is

\begin{equation}
    w^D(r_{ij}) = \left [w^R(r_{ij}) \right ]^2 = \begin{cases}
        \left (1 - \frac{r_{ij}}{r_c} \right )^2, & r_{ij} \leq r_c \\
        0, & r_{ij} > r_c
    \end{cases}\, .
\end{equation}

Both forces are controlled by strength coefficients, namely $\sigma$ for the random force
and $\gamma$ for the drag force. The fluctuation dissipation theorem connects these
two parameters with the thermal Energy $k_B T$ and yields

\begin{equation}
    \sigma^2 = 2 \gamma k_B T.
\end{equation}

Defining $\mathbf{v}_{ij} = |\mathbf{v}_i - \mathbf{v}_j|$ as the 
relative velocity between two particles
and $\alpha_{ij}$ as 
a random number with zero mean and unit variance 
results in a final expression for both forces

\begin{align}
    \mathbf{F}^D_{ij} &= -\gamma w^D(r_{ij})(\hat{\mathbf{r}}_{ij}\cdot \mathbf{v}_{ij})
    \hat{\mathbf{r}}_{ij} \\
    \mathbf{F}^R_{ij} &= \sigma w^R(r_{ij}) \alpha_{ij} \Delta t^{-1/2} \hat{\mathbf{r}}_{ij}\, .
\end{align}

Notably a factor of $\Delta t^{-1/2}$ appears in the random force. This
follows the argument of Groot and Warren (\textbf{cite here}) as this 
corrects the time dependency of displacement to the known expression 
$\left < x^2\right >\propto t$.

To summarize to fully describe DPD interaction, the only parameters to determine are the 
repulsion parameters $A_{ij}$ and the drag coefficient $\gamma$ which in turn fixes the 
random force parameter $\sigma$. This step will be further discussed in \cref{params}.

\section{Electrostatic Interaction}\label{coulomb}

The DPD force in \cref{dpd} does not differentiate between a charged or an uncharged bead, meaning 
some kind of coulomb interaction has to be added to the force field, 
or the repulsion have to be adapted correspondingly,
to the force field. Crucially a simple 
coulomb potential is an unfitting choice because it would destroy the soft property of the
potential at $r=0$. To solve this 
a exponentially smeared charge distribution resulting in a Slater potential is employed 
. This is following the work of Wang and Hernandez (\textbf{cite here})
who also introduce a factor $\beta$ which controls the strength of the
smearing effect. The charge distribution then reads

\begin{equation}
    \rho(r) = \frac{q\beta^2}{\pi r} \exp(-2\beta r)\, .
\end{equation}

Using earlier works from Gonzales-Melchor et al (\textbf{cite here})
this factor and its inverse $\lambda$ which is used in the LAMMPS implementation 
of the potential was calculated as:

    
\[
    \beta = 0.844, \quad \lambda = 1/\beta = 1.185\, .
\]

With this and a conversion constant $C$ the coulomb interaction follows

\begin{align}
    \mathbf{F}^E_{ij} = \frac{C q_i q_j}{4 \pi r_{ij}}[1 - \exp(-2\beta r_{ij})
    (1+2\beta r_{ij}(1+\beta r_{ij}))]\hat{\mathbf{r}}_{ij}\, .
\end{align}


\section{Intramolecular Interactions}\label{intermol}

\cref{dpd} and \cref{coulomb} define the interaction between non-bonded beads. Beyond that the beads are also  
bonded inside a molecule. Like described in \cref{CG_models} the corresponding information are found in
the \path{martini_v_2.0_POPC.itp} file. For POPC there are only bond and angle interactions.  

For the bond interaction the potential is a simple harmonic potential, which is implemented 
in LAMMPS as

\begin{equation}
    E_{ij} = K^b(r_{ij} - r^0_{ij})^2. \label{harmonic}\, ,
\end{equation}

where $K_b$ is a spring constant in the unit of force that contains the usual factor of $1/2$ and $r^0_{ij}$ is the equilibrium
bond distance of the bead pair. This works similarly for the angle using the potential

\begin{equation}
    E_{ijk} = K^a[\cos(\theta_{ijk} - \cos(\theta^0_{ijk}))]^2\, , \label{cosine/squared}
\end{equation}


with $K_a$ being a constant in the unit of energy including the factor of $1/2$ and similarly
to before $\theta^0_{ijk}$ the equilibrium angle between the three beads. Though the choice 
is given by the MARTINI database it is interesting to ask why potential of choice is 
not simply harmonic for the angles. The simple answer is that the potential in \cref{cosine/squared}
is much more computationally efficient because the calculation of a cosine does not include the $\arccos(\theta)$
function, which is costly and not robust.

\section{DPD units}\label{dpd_units}

Molecular dynamics simulations often employ reduced units.
Meaning there are reference parameters which are used to make every physical quantity
dimensionless. For MARTINI these are usually the parameters of the Lennard-Jones potential
$\epsilon$ and $\sigma$ as in \cref{LJ_pot}. But since DPD has a very different 
force and thus potential one needs different reference parameters. For the 
length parameter, the work of Wang and Hernandez (\textbf{cite here}) is followed who 
choose the length scale as the length of a cube containing three CG water beads. The number three 
is not arbitrarily chosen and follows the early
work of Groot and Warren 
(\textbf{cite here}) who used the number density $\rho=3$ beads per unit of volume. 
In MARTINI every water bead contains $N_w = 4$ molecules and
the volume of one water molecule $V_w$ can be calculated by
using the molar mass $M=18.02 \si{\g\per\mol}$,
the density $\rho_w = 1 \si{\g\per\cubic\centi\meter}$ and the avogadro
constant $N_A = 6.022 \cdot 10^{23} 
\text{molecules}/\text{mol}$

\begin{equation}
    V_w = \frac{M}{\rho_w N_A} = 2.99 \cdot 10^{-23} \si{\cubic\centi\m} 
    = 29.9 \si{\cubic\angstrom}\, .
\end{equation}

The DPD length follows directly

\begin{equation}
    r_{\text{ref}} = \sqrt[3]{\rho \cdot N_w \cdot V_w} = 7.11 \si{\angstrom}\, ,
\end{equation}

which from now on can be used to convert every length into DPD units dividing it.

For an energy scale the chosen value is much simpler. The reference temperature was defined as
$T_{\text{ref}}=298.15 \si{\K}$ and used the thermal energy as a reference resulting in 

\begin{equation}
    \epsilon_{\text{ref}} = k_{\mathrm{B}} T_{\text{ref}} = 4.12 \cdot 10^{-21} \si{\J} = 
    25.7 \si{\milli\eV}\, .
\end{equation}

Similarly to the length scale this value can now be used to convert different energies into DPD
units. Furthermore, the combination of scales allows conversion of every physical quantity. 


\section{Determination of DPD bonded parameters}\label{params}

As \cref{CG_models} suggests the equilibrium parameters are found in the \path{martini_v_2.0_POPC.itp}
file

\begin{align*}
    r^0_1 &= 0.47 \si{\nano\m}\, , & k^b_1 &= 1250 \si{\kilo\J\per\mol\per\nano\m\squared} \\
    r^0_2 &= 0.37 \si{\nano\m}\, , & k^b_2 &= 1250 \si{\kilo\J\per\mol\per\nano\m\squared}\, .
\end{align*}

While the bond length can simply be divided by the reference length from 
\cref{dpd_units}, the spring constant has to be handled with more care. As described in \cref{intermol} 
the MARTINI definition still has a factor $1/2$ in front of 
the harmonic potential which is absorbed in the LAMMPS implementation, 
see \cref{harmonic}. This yields

\begin{equation}
    \hat{K}^b_i = \frac{1}{2} k^b_i \frac{r_{\text{ref}}^2}{e_{\text{ref}}N_A}\, ,
\end{equation}

resulting in the following dimensionless values


\begin{align*}
\hat{r}^0_1 &= 0.661\, , & \hat{K}^b_1 &= 127.5\\
\hat{r}^0_2 &= 0.520\, , & \hat{K}^b_2 &= 127.5\, .
\end{align*}

This calculation for the angle parameters only differs in the units of the equilibrium parameters

\begin{align*}
\theta^0_1 &= 180^\circ\, , & k^a_1 &= 25 \si{\kilo\J\per\mol}\\
\theta^0_2 &= 120^\circ\, , & k^a_2 &= 25 \si{\kilo\J\per\mol}\\
\theta^0_3 &= 120^\circ\, , & k^a_3 &= 45 \si{\kilo\J\per\mol}\, .
\end{align*}

Similarly to before the factor of $1/2$ has to be included

\begin{equation}
    \hat{K}^a_i = \frac{1}{2}  \frac{k^a_i}{e_{\text{ref}}N_A}\, .
\end{equation}

Substituting the values yields the angle parameters


\begin{align*}
\hat{\theta}^0_1 &= 180^\circ\, , & \hat{k}^a_1 &= 5.04\\
\hat{\theta}^0_2 &= 120^\circ\, , & \hat{k}^a_2 &= 5.04\\
\hat{\theta}^0_3 &= 120^\circ\, , & \hat{k}^a_3 &= 9.08\, .
\end{align*}

\section{Determination of DPD Charge}

Introducing a length and energy sclaes also has an effect on the charge quantization. To adjust 
the charge the energy of a point charge in reals units with the corresponding energy in DPD units 
yielding 

\begin{align}
    &E = \frac{q^2}{4\pi \epsilon_0 r} \stackrel{!}{=} \hat{E} e_{\text{ref}} = 
    \frac{\hat{q}^2}{\hat{r}} e_{\text{ref}}\\
    &\hat{q} = \frac{q}{\sqrt{4\pi \epsilon_0 r_{\text{ref}}e_{\text{ref}}}} = 8.86\, .
\end{align}

Defining this charge means that a charge of one elemental charge in real units corresponds 
to a DPD charge of $\hat{q} = 8.86$.

\section{Determination of DPD force parameters}

The repulsion parameters $A_{ij}$ for the DPD force are the core of the force field and 
determine the maximum repulsion
at $r_{ij} =0$. In addition to that the drag parameter $\gamma$ has to be determined.
The latter follows from the early work of Groot and Warren (\textbf{cite here}) who
found at $\hat{T} =1$ the optimal value for the random force parameter was $\hat{\sigma} = 3$.
Employing the connection given by the fluctuation dissipation theorem yields a values of
 
\[
\hat{\gamma} = 4.5\, ,
\]

where $k_{\text{B}}T$ was set to unity. The $\gamma$ value is a fix parameter in the DPD 
force field utilized by the LAMMPS package. Changing the desired temperature of the DPD force 
adjusts the $\sigma$ parameter according to the fluctuation dissipation theorem.

The repulsion parameters used in the simulation were derived by Wang and Hernandez (\textbf{cite here}).
The values are shown in \cref{dpd_params}. The derivation method is described below.

The baseline is the repulsion parameter between two water beads W-W, called $a_{ww}$. 
This is done by matching the inverse compressibility $\kappa^{-1}$ from experimental data 
to the corresponding value of $a_{ww}$. Similarly to the MARTINI model where the 
$\sigma$ parameter of W-W interaction is used for every other same bead interaction this value is 
also adapted for every same bead interaction here as well ,meaning $a_{AA} = a_{ww}$ for all 
beads. The measured value was 

\[
a_{ww} = 102.067\, .
\]


\begin{table}[t]
\centering
\begin{tabular}{|cc|cc|}
\hline
Bead pair & $A_{ij}$ & Bead pair & $A_{ij}$\\
\hline 
W - C1     & 130.616 & Q0 - Q0    & 102.067 \\ 
W - C3     & 121.734 & Q0 - C1    & 121.099 \\ 
W - Na     & 108.411 & Q0 - C3    & 117.293 \\ 
W - Q0     & 84.938  & Q0 - Na    & 98.895  \\ 
W - Qa     & 94.454  & Na - Na    & 102.067 \\ 
Qa - Qa    & 102.067 & Na - C1    & 115.390 \\ 
Qa - C1    & 130.616 & Na - C3    & 115.390 \\ 
Qa - C3    & 126.809 & C3 - C3    & 102.067 \\ 
Qa - Na    & 108.411 & C3 - C1    & 102.067 \\ 
Qa - Q0    & 98.895  & C1 - C1    & 102.067 \\ 

\hline
\end{tabular}
\caption{DPD force repulsion parameter $a_{ij}$ for every bead pair. The same bead interaction
are the same as the water-water interaction which was derived by matching simulation data to
experimental data. Other parameters were derived using the Flory-Huggins parameter and the 
interaction paramters of the MARTINI model.}
\label{dpd_params}
\end{table}

From there the interaction of two arbitrary beads can be expressed as 

\begin{equation}
    a_{AB} = a_{ww} + \Delta a_{AB}\, ,
\end{equation}

with $\Delta a_{AB}$ being connected to the Flory-Huggins parameter 

\begin{equation}
    \chi_{AB} = \lambda \Delta a_{AB}\, ,
\end{equation}

using a fitting parameter $\lambda$ that has to be determined.
Earlier works by Groot and Rabone (\textbf{cite here}) indicate that the Flory-Huggins 
parameter scales linear with the size of the beads which in this system is always $N_m = 4$ 
molecules per bead, meaning 

\begin{equation}
    \chi_{AB} = \bar{\chi}_{AB} N_m\, .
\end{equation}

This results in a final formula for the interaction parameter

\begin{equation}
    a_{AB} = a_{ww} + \bar{\chi}_{AB} N_m/\lambda\, . \label{final_form_dpd}
\end{equation}

To determine the $\lambda$ fitting parameter a box seperately filled with two different beads $A$ and $B$
is simulated. After equilibration the volume fraction is measured. This method is repeated for 
different values of $\Delta a_{AB}$. For each simulation run the Flory-Huggins paremter is 
calculated with

\begin{equation}
    \chi = \frac{\ln[(1-\phi)/\phi]}{1 - 2\phi}
\end{equation}

and plotted against the parameter $\Delta a_{AB}$. The slope of the resulting linear graph 
is the desired fitting parameter 

\[
\lambda = 0.28\, .
\]

Wang and Hernandez (\textbf{cite here}) found a formula which uses a baseline interaction matrix
$\epsilon_{AB}$ to calculate the size dependant Flory-Hugigns parameter.

\begin{equation}
    \bar{\chi}_{AB} = \chi_{AB} = \frac{z}{k_{\mathrm{B}}T} \left[\epsilon_{AB} - 
    \frac{1}{2} (\epsilon_{AA} + \epsilon_{BB}) \right] \, ,
\end{equation}

using $z$ as the effective number of nearest neighbors.

The employed database 
is the MARTINI model. The calculated values can be substituted into \cref{final_form_dpd} and 
result in the values shown in \cref{dpd_params}.

























\end{document}
